% Options for packages loaded elsewhere
% Options for packages loaded elsewhere
\PassOptionsToPackage{unicode}{hyperref}
\PassOptionsToPackage{hyphens}{url}
%
\documentclass[
  english,
  russian,
  12pt,
  a4paper,
  DIV=11,
  numbers=noendperiod]{scrreprt}
\usepackage{xcolor}
\usepackage{amsmath,amssymb}
\setcounter{secnumdepth}{5}
\usepackage{iftex}
\ifPDFTeX
  \usepackage[T1]{fontenc}
  \usepackage[utf8]{inputenc}
  \usepackage{textcomp} % provide euro and other symbols
\else % if luatex or xetex
  \usepackage{unicode-math} % this also loads fontspec
  \defaultfontfeatures{Scale=MatchLowercase}
  \defaultfontfeatures[\rmfamily]{Ligatures=TeX,Scale=1}
\fi
\usepackage{lmodern}
\ifPDFTeX\else
  % xetex/luatex font selection
\fi
% Use upquote if available, for straight quotes in verbatim environments
\IfFileExists{upquote.sty}{\usepackage{upquote}}{}
\IfFileExists{microtype.sty}{% use microtype if available
  \usepackage[]{microtype}
  \UseMicrotypeSet[protrusion]{basicmath} % disable protrusion for tt fonts
}{}
\usepackage{setspace}
% Make \paragraph and \subparagraph free-standing
\makeatletter
\ifx\paragraph\undefined\else
  \let\oldparagraph\paragraph
  \renewcommand{\paragraph}{
    \@ifstar
      \xxxParagraphStar
      \xxxParagraphNoStar
  }
  \newcommand{\xxxParagraphStar}[1]{\oldparagraph*{#1}\mbox{}}
  \newcommand{\xxxParagraphNoStar}[1]{\oldparagraph{#1}\mbox{}}
\fi
\ifx\subparagraph\undefined\else
  \let\oldsubparagraph\subparagraph
  \renewcommand{\subparagraph}{
    \@ifstar
      \xxxSubParagraphStar
      \xxxSubParagraphNoStar
  }
  \newcommand{\xxxSubParagraphStar}[1]{\oldsubparagraph*{#1}\mbox{}}
  \newcommand{\xxxSubParagraphNoStar}[1]{\oldsubparagraph{#1}\mbox{}}
\fi
\makeatother


\usepackage{longtable,booktabs,array}
\usepackage{calc} % for calculating minipage widths
% Correct order of tables after \paragraph or \subparagraph
\usepackage{etoolbox}
\makeatletter
\patchcmd\longtable{\par}{\if@noskipsec\mbox{}\fi\par}{}{}
\makeatother
% Allow footnotes in longtable head/foot
\IfFileExists{footnotehyper.sty}{\usepackage{footnotehyper}}{\usepackage{footnote}}
\makesavenoteenv{longtable}
\usepackage{graphicx}
\makeatletter
\newsavebox\pandoc@box
\newcommand*\pandocbounded[1]{% scales image to fit in text height/width
  \sbox\pandoc@box{#1}%
  \Gscale@div\@tempa{\textheight}{\dimexpr\ht\pandoc@box+\dp\pandoc@box\relax}%
  \Gscale@div\@tempb{\linewidth}{\wd\pandoc@box}%
  \ifdim\@tempb\p@<\@tempa\p@\let\@tempa\@tempb\fi% select the smaller of both
  \ifdim\@tempa\p@<\p@\scalebox{\@tempa}{\usebox\pandoc@box}%
  \else\usebox{\pandoc@box}%
  \fi%
}
% Set default figure placement to htbp
\def\fps@figure{htbp}
\makeatother



\ifLuaTeX
\usepackage[bidi=basic,provide=*]{babel}
\else
\usepackage[bidi=default,provide=*]{babel}
\fi
% get rid of language-specific shorthands (see #6817):
\let\LanguageShortHands\languageshorthands
\def\languageshorthands#1{}


\setlength{\emergencystretch}{3em} % prevent overfull lines

\providecommand{\tightlist}{%
  \setlength{\itemsep}{0pt}\setlength{\parskip}{0pt}}



 
\usepackage[backend=biber,langhook=extras,autolang=other*]{biblatex}
\addbibresource{bib/cite.bib}

\usepackage[]{csquotes}

\KOMAoption{captions}{tableheading}
\usepackage{indentfirst}
\usepackage{float}
\floatplacement{figure}{H}
\usepackage{libertine}
\makeatletter
\@ifpackageloaded{caption}{}{\usepackage{caption}}
\AtBeginDocument{%
\ifdefined\contentsname
  \renewcommand*\contentsname{Содержание}
\else
  \newcommand\contentsname{Содержание}
\fi
\ifdefined\listfigurename
  \renewcommand*\listfigurename{Список иллюстраций}
\else
  \newcommand\listfigurename{Список иллюстраций}
\fi
\ifdefined\listtablename
  \renewcommand*\listtablename{Список таблиц}
\else
  \newcommand\listtablename{Список таблиц}
\fi
\ifdefined\figurename
  \renewcommand*\figurename{Рисунок}
\else
  \newcommand\figurename{Рисунок}
\fi
\ifdefined\tablename
  \renewcommand*\tablename{Таблица}
\else
  \newcommand\tablename{Таблица}
\fi
}
\@ifpackageloaded{float}{}{\usepackage{float}}
\floatstyle{ruled}
\@ifundefined{c@chapter}{\newfloat{codelisting}{h}{lop}}{\newfloat{codelisting}{h}{lop}[chapter]}
\floatname{codelisting}{Список}
\newcommand*\listoflistings{\listof{codelisting}{Листинги}}
\makeatother
\makeatletter
\makeatother
\makeatletter
\@ifpackageloaded{caption}{}{\usepackage{caption}}
\@ifpackageloaded{subcaption}{}{\usepackage{subcaption}}
\makeatother
\usepackage{bookmark}
\IfFileExists{xurl.sty}{\usepackage{xurl}}{} % add URL line breaks if available
\urlstyle{same}
\hypersetup{
  pdftitle={Отчет по лабораторной работе №6},
  pdfauthor={Садыков Ильдар Ильфатович},
  pdflang={ru-RU},
  hidelinks,
  pdfcreator={LaTeX via pandoc}}


\title{Отчет по лабораторной работе №6}
\usepackage{etoolbox}
\makeatletter
\providecommand{\subtitle}[1]{% add subtitle to \maketitle
  \apptocmd{\@title}{\par {\large #1 \par}}{}{}
}
\makeatother
\subtitle{Архитектура компьютера}
\author{Садыков Ильдар Ильфатович}
\date{}
\begin{document}
\maketitle

\renewcommand*\contentsname{Содержание}
{
\setcounter{tocdepth}{1}
\tableofcontents
}
\listoffigures
\listoftables

\setstretch{1.5}
\chapter{Цель
работы}\label{ux446ux435ux43bux44c-ux440ux430ux431ux43eux442ux44b}

Освоение арифметических инструкций языка ассемблера NASM.

\chapter{Задание}\label{ux437ux430ux434ux430ux43dux438ux435}

\begin{enumerate}
\def\labelenumi{\arabic{enumi}.}
\tightlist
\item
  Изучить арифметические инструкции языка ассемблера NASM.
\item
  Освоить работу с символьными и численными данными в NASM.
\item
  Написать программы для выполнения арифметических операций.
\item
  Выполнить задание для самостоятельной работы по вычислению значения
  функции.
\end{enumerate}

\chapter{Выполнение лабораторной
работы}\label{ux432ux44bux43fux43eux43bux43dux435ux43dux438ux435-ux43bux430ux431ux43eux440ux430ux442ux43eux440ux43dux43eux439-ux440ux430ux431ux43eux442ux44b}

\section{\texorpdfstring{\textbf{Программа вывода значения регистра
eax:}}{Программа вывода значения регистра eax:}}\label{ux43fux440ux43eux433ux440ux430ux43cux43cux430-ux432ux44bux432ux43eux434ux430-ux437ux43dux430ux447ux435ux43dux438ux44f-ux440ux435ux433ux438ux441ux442ux440ux430-eax}

Введем код программы для сложения символов \enquote*{6} и \enquote*{4}
(рис.~\ref{fig-001}).

\begin{figure}

\centering{

\pandocbounded{\includegraphics[keepaspectratio]{image/1.1.png}}

}

\caption{\label{fig-001}Код и первое исполнение программы lab6-1.asm}

\end{figure}%

\textbf{Результат:} Программа вывела символ \enquote*{j}, так как
сложила ASCII-коды символов \enquote*{6} (54) и \enquote*{4} (52),
получив 106 - код символа \enquote*{j}.

\section{\texorpdfstring{\textbf{Модификация программы с
числами:}}{Модификация программы с числами:}}\label{ux43cux43eux434ux438ux444ux438ux43aux430ux446ux438ux44f-ux43fux440ux43eux433ux440ux430ux43cux43cux44b-ux441-ux447ux438ux441ux43bux430ux43cux438}

Изменим программу, используя числа вместо символов (рис.~\ref{fig-002}).

\begin{figure}

\centering{

\pandocbounded{\includegraphics[keepaspectratio]{image/1.2.png}}

}

\caption{\label{fig-002}Измененный код lab6-1.asm}

\end{figure}%

\begin{verbatim}
**Результат:** Программа вывела символ с кодом 10 ( или же \n по таблице ascii).
\end{verbatim}

\section{\texorpdfstring{\textbf{Программа с преобразованием в число:
}}{Программа с преобразованием в число: }}\label{ux43fux440ux43eux433ux440ux430ux43cux43cux430-ux441-ux43fux440ux435ux43eux431ux440ux430ux437ux43eux432ux430ux43dux438ux435ux43c-ux432-ux447ux438ux441ux43bux43e}

Создадим файл \texttt{lab6-2.asm} с использованием функции
преобразования, а после исправим код и занова запустим
(рис.~\ref{fig-003}).

\begin{figure}

\centering{

\pandocbounded{\includegraphics[keepaspectratio]{image/1.3.png}}

}

\caption{\label{fig-003}Создание и выполнение lab6-2.asm}

\end{figure}%

\textbf{Результат:} Программа вывела число 106 - сумму ASCII-кодов
символов. А после корректировки верный ввывод - число 10.

При изменении iprintLF на iprint, после вывода ответа строка не перешла
последующую.

\section{\texorpdfstring{\textbf{Вычисление
выражений:}}{Вычисление выражений:}}\label{ux432ux44bux447ux438ux441ux43bux435ux43dux438ux435-ux432ux44bux440ux430ux436ux435ux43dux438ux439}

Напишем программу для вычисления выражения f(x)= (5*2+3)/3
(рис.~\ref{fig-004}).

\begin{figure}

\centering{

\pandocbounded{\includegraphics[keepaspectratio]{image/1.4.png}}

}

\caption{\label{fig-004}Запуск и вывод lab6-3}

\end{figure}%

\section{\texorpdfstring{\textbf{Изменненое
выражение:}}{Изменненое выражение:}}\label{ux438ux437ux43cux435ux43dux43dux435ux43dux43eux435-ux432ux44bux440ux430ux436ux435ux43dux438ux435}

Напишем программу для вычисления выражения f(x)= (4*6+2)/5
(рис.~\ref{fig-005}).

\begin{figure}

\centering{

\pandocbounded{\includegraphics[keepaspectratio]{image/1.5.png}}

}

\caption{\label{fig-005}Запус измененной программы lab6-3}

\end{figure}%

\section{\texorpdfstring{\textbf{Создание и запуск программы
variant.asm:}}{Создание и запуск программы variant.asm:}}\label{ux441ux43eux437ux434ux430ux43dux438ux435-ux438-ux437ux430ux43fux443ux441ux43a-ux43fux440ux43eux433ux440ux430ux43cux43cux44b-variant.asm}

Создадим программу для вычисления варианта по номеру студенческого
билета (рис.~\ref{fig-006}).

\begin{figure}

\centering{

\pandocbounded{\includegraphics[keepaspectratio]{image/1.6.png}}

}

\caption{\label{fig-006}Создание и запуск variant.asm}

\end{figure}%

\chapter{\texorpdfstring{\textbf{Ответы на
вопросы:}}{Ответы на вопросы:}}\label{ux43eux442ux432ux435ux442ux44b-ux43dux430-ux432ux43eux43fux440ux43eux441ux44b}

\begin{enumerate}
\def\labelenumi{\arabic{enumi}.}
\tightlist
\item
  За вывод сообщения \enquote*{Ваш вариант:} отвечает строка:
  \texttt{mov\ eax,rem} и \texttt{call\ sprint}
\item
  Инструкции \texttt{mov\ ecx,\ x}, \texttt{mov\ edx,\ 80},
  \texttt{call\ sread} используются для ввода строки с клавиатуры
\item
  \texttt{call\ atoi} используется для преобразования ASCII-строки в
  число
\item
  За вычисление варианта отвечают строки с \texttt{div\ ebx} и
  \texttt{inc\ edx}
\item
  Остаток от деления записывается в регистр EDX
\item
  \texttt{inc\ edx} используется для увеличения остатка на 1 (так как
  варианты нумеруются с 1)
\item
  За вывод результата отвечают строки: \texttt{mov\ eax,rem},
  \texttt{call\ sprint}, \texttt{mov\ eax,edx}, \texttt{call\ iprintLF}
\end{enumerate}

\chapter{Задание для самостоятельной
работы}\label{ux437ux430ux434ux430ux43dux438ux435-ux434ux43bux44f-ux441ux430ux43cux43eux441ux442ux43eux44fux442ux435ux43bux44cux43dux43eux439-ux440ux430ux431ux43eux442ux44b}

\section{\texorpdfstring{\textbf{Программа вычисления функции
f(x):}}{Программа вычисления функции f(x):}}\label{ux43fux440ux43eux433ux440ux430ux43cux43cux430-ux432ux44bux447ux438ux441ux43bux435ux43dux438ux44f-ux444ux443ux43dux43aux446ux438ux438-fx}

Для самостоятельной работы необходимо написать программу вычисления
выражения согласно варианту (в моем случае по номеру 7).

Напишем программу для вычисления выражения f(x)= 5(x-1)\^{}2
(рис.~\ref{fig-007}).

\begin{figure}

\centering{

\pandocbounded{\includegraphics[keepaspectratio]{image/2.1.png}}

}

\caption{\label{fig-007}Запуск кода и провека с со зачениями из таблицы
(Все верно!)}

\end{figure}%

\chapter{Выводы}\label{ux432ux44bux432ux43eux434ux44b}

В ходе выполнения лабораторной работы были успешно освоены
арифметические инструкции языка ассемблера NASM. Были изучены основные
арифметические операции (сложение, вычитание, умножение, деление), а
также особенности работы с символьными и численными данными. Особое
внимание было уделено преобразованию между ASCII-символами и числами с
использованием функций из файла in\_out.asm. Были написаны и
протестированы программы для вычисления арифметических выражений и
определения варианта задания по номеру студенческого билета. Полученные
навыки позволяют эффективно работать с арифметическими операциями на
уровне ассемблера.

\printbibliography[heading=none]





\end{document}
